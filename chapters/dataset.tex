The data set we were working with consisted of pages which had been scraped from the dark web in ElasticSearch. This chapter will describe the dark web, the ElasticSearch database, and the structure of the stored pages.

The internet consists of networks from all over the world. The networks are connected and together they comprise a global network. The Internet can be divided into the clear web\ref{clearWeb}, and the deep net\ref{deepWeb}.

\section{Clear web} \label{clearWeb}
The clear web, or Clearnet, represents approximately 4\% of the Internet. The content on the clearnet is indexed by search engines and is publicly accessible. The users are identified by their IP addresses and are usually not anonymous unless some privacy tools are used.

\section{Deep web} \label{deebWeb}
The deep web, or hidden web, represents approximately 96\% of the Internet. The content is not indexed by search engines and is not accessible publicly. Emails or medical information are an example of the deep web. A portion of the deep web is the dark web, also called darknet. 

\subsection{Dark web} \label{darkWeb}
The dark web constitutes about 6.25\% of the deep web. Tor \cite{torIntro} or I2P \cite{i2pIntro} are some of the networks comprising the darknet. The dark web is accessible via special browser, e.g. Tor or I2P. Both exampled browsers provide anonymous access to the as well as to the darknet. 

The dark web is used for activities like accessing or publishing illegal or censored material, e.g. the Bible, whistle-blower secrets, or child porn. It is also used for trading with illegal goods, such as drugs or guns. Another way to use the dark web is to offer or order illegal services, for instance money laundering, hacking or murder.

\section{The data set} \label{dataSet}
The dark web pages were acquired via web scraping\footnote{A method used for collecting data from web pages with automated software rather than doing so manually.}. The scraped pages belonged to two different networks - I2P  and Tor. Both networks provide anonymity for the user. The total number of pages is 221,844. Of those are 212,851 Tor pages and 8,993 I2P pages. The total number of unique domains is 5,178 of which 4,912 are from the Tor network and 266 are from the I2P network. 

The fields of a page entry will be described in the following list. Each description will include a concrete example from the database. Fields beginning with an underscore are assigned to every document implicitly.
\begin {description}
	\item[\_id] is a unique identifier. For example\\ \textit{2d622b6fba6f203d790fedbb4f47963e2366c7fd}. This field is indexed. 
	\item[\_index] informs about the collection the document belongs to. For example \textit{tor}. 
	\item[\_type] determines the type of the document. For example \textit{\_doc}. This field is indexed. 
	\item[content] is the actual content of the page. For example \textit{Purple Kush – 10g – WackyWeed Menu    * Home   * Contact us   * About us }.
	\item[content\_type] describes the type of the content. For example \textit{text/html; charset=UTF-8}.
	\item[domain] of the url address. For example \textit{wacky2yx73r2bjys.onion}.
	\item[h1] is the text with the h1 style. For example \textit{Purple Kush – 10g}.
	\item[links] to other pages this page links to. For example \textit{\{\\
  "link": "http://wacky2yx73r2bjys.onion/",\\
  "link\_name": "Home"\\
\}}.
	\item[raw\_text] is similar to \textit{content}. It additionally may contain formatting elements such as \textit{\textbackslash n}. For example \textit{Purple Kush – 10g – WackyWeed Menu\textbackslash n\textbackslash n  * Home\textbackslash n  * Contact us\textbackslash n  * About us\textbackslash n }.
	\item[raw\_title] is similar to title. It additionally may contain formatting elements. For example \textit{Purple Kush – 10g – WackyWeed}
	\item[raw\_url] is the same as \textit{url}.
	\item[title] of the page. For example \textit{Purple Kush – 10g – WackyWeed}.
	\item[updated\_on] depicts the time when the document was last updated. For example \textit{2019-10-22T19:41:09}.
	\item[url] address of the page. For example \\ \textit{http://wacky2yx73r2bjys.onion/?product=purple-kush-10g}.
\end{description}