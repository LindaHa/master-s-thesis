\label{conclusion}
\section{Summary}
The goal of this thesis was to create a tool for the categorization, and the visualization of scraped pages, along with an option to analyse the pages in more detail.

We first analysed the scraped pages stored in Elasticsearch in Chapter \ref{datasetAnalysis}. In light of the size of the data set at hand we adopted supervised ML for automated categorization. Therefore, a labeled learning data set was built, and improved iteratively. We constructed a CNN and trained a classification model on the learning data set of each iteration. We discovered commonly pre-trained embeddings do not improve the learning results. We compared the learning results on the individual data sets in Section \ref{classificationEvaluation}. The model achieved a prediction accuracy of 88.2\%. This accuracy was compared to accuracies achieved by models trained for similar tasks. We concluded our accuracy to be satisfying despite being worse.

 We assumed a web graph to be an appropriate way to represent the pages. However, the data set size proved also problematic in the visualization of the web graph as mentioned in Chapter \ref{clustering}. We observed the structure of the web graph to be suitable for community detection. We then considered the LA and the LeA algorithm and applied both. The LA detected more of total and decomposable communities. On the other hand, some of the sub-communities detected by this algorithm were disconnected. The LeA did not produce any disconnected sub-communities. Additionaly, the LeA performed significantly faster than LA as was described in Section \ref{clusteringEvaluation}.

A BE with a Rest API was created. The BE categorized the pages via the classification model. The clustering of the pages took also place in the BE. Additionaly, we created a FE for the displaying of the web graph and page details. 

 To our knowledge the web application created as part of the thesis is the only tool publicly available to perform all of the above listed requirements.

\section{Future work}
\subsection{Categorization}

Presently our classification model predicts categories with 88.2\% accuracy. Creating with more pages in certain categories may improve the accuracy. 

Currently the classification supports only the English language. A portion of the data set content is therefore discarded, or categorized incorrectly or as category \textit{Other}. Supporting other languages may improve the classification performance. 

\subsection{User interface}

At the current state the category colours are not adjustable. Adding a colour picker would provide the user the option to customize the colours to their liking.

The selection of the clustering algorithm is also not changeable on the UI level. The introduction of a switch for the swapping between the LA or the LeA would present another improvement.
\subsection{API}

The BE provides the option to choose between the LA and the LeA for clustering. However, the API does not mirror this option. An adjustment of the API for this purpose would prove useful. 

Page or community details are currently returned in JSON format. We think supporting other formats, such as CSV, would enhance the application.
