\label{datasetAnalysis}
The data set we were working with consisted of pages which had been scraped\footnote{Scraping is a method used for collecting data from web pages with automated software rather than doing so manually.} from the dark web and stored in Elasticsearch (ES) \cite{elasticSearch}. This chapter will describe the clear web, deep web, and the dark web, the Elasticsearch database, and the structure of the stored pages.

The Internet consists of networks from all over the world. The networks are connected and together they comprise a global network. The Internet can be divided into the clear web \ref{clearWeb}, and the deep web~\ref{deepWeb} \cite{internetStructure}.

\section{Clear web} \label{clearWeb}
The content of the clear web is indexed by search engines and is publicly accessible. The users are identified by their IP addresses and are usually not anonymous unless some privacy tools are used. The most used browsers worldwide according to market share are Chrome, Safari and Firefox \cite{browserMarketShare}. The size of the clear web is difficult to determine. However, it is estimated to contain about 5.93 billion pages as of March 2020 \cite{clearWebSize}.

\section{Deep web} \label{deepWeb}
The content of the deep web is not indexed by search engines and is not accessible publicly. Emails or medical information are examples of resources in the deep web. A user needs to be authorized to access this data. It is more problematic to measure the size of the deep web because of the fact the content is not indexed. The size was estimated to be 4,000 to 5,000 times larger than the clear web \cite{deepWebSize}. 

A portion of the deep web is the \textit{dark web}, also called darknet. 

\subsection{Dark web} \label{darkWeb}
The dark web provides anonymized services and user identities. It is therefore leveraged for activities like accessing or publishing illegal or censored material, e.g. the Bible, whistle-blower secrets, or child porn. It is also used for trading with illegal goods, such as drugs or guns. Another way to exploit the dark web is to offer or order illegal services, for instance money laundering, hacking or murder \cite{theDarkNet}. The onion router (Tor) \ref{tor} or the Invisible Internet Project (I2P) \ref{I2P} are two of the networks comprising the darknet. 

\subsection{Tor} \label{tor}
The Tor network \cite{torIntro} is accessible through the Tor browser or Tor proxy. Tor makes use of \textit{onion routing} (ORing) \cite{onionRouting}. ORing describes routing where each node, except for the originator and the target node, knows only its predecessor and successor. The originator chooses a list of nodes from which it creates a circuit between itself and the target node. Nodes in this circuit are determining their successor leveraging public-key cryptography. For example, \textit{A} is the originator and \textit{D} the target node. \textit{A} creates a circuit consisting of itself and of nodes \textit{B}, \textit{C} and \textit{D}. \textit{A} creates a fixed-sized cell with encrypted information about the circuit nodes with a message using their respective public keys. \textit{A} sends this cell to \textit{B}. \textit{B} removes one layer of the cell using its private key in order to get the address of its successor, in this case \textit{C}. \textit{B} cannot remove an additional layer to the successor of \textit{C} because it lacks the private key of \textit{C}. \textit{B} sends the cell to \textit{C}. \textit{C} removes another layer and sends the cell to \textit{D}. \textit{D} removes the final layer and reads the information. Now \textit{D} is able to communicate with \textit{A} through \textit{D} and \textit{C} with only knowing about itself and \textit{D}. Determining the original source and target node is therefore difficult. ORing was introduced in the 1990s to ensure privacy. 

\subsection{I2P} \label{I2P}
I2P \cite{i2pIntro} is another network of the dark web which anonymizes its traffic. It is accessible through the I2P browser. In contrast to Tor, communication in I2P is based on \textit{garlic routing} (GRing) \cite{garlicRouting}. GRing is described as an extension of ORing. The established tunnels between two nodes are unidirectional, meaning different tunnels are used for outgoing and incoming messages. Another difference from ORing is the bundling of messages. An individual message is called a \textit{clove}\footnote{Michael Freedman, who defined garlic routing, called cloves \textit{bulbs}.}. Each clove contains their own delivery instructions. These instructions are exposed at the target node. Cloves are bundled into a \text{garlic-message}. The bundling of cloves ensures more secure and efficient communication. 

Both exampled browsers provide anonymous access to the clear web as well as to the dark web. Both are open-source and free to use.

\section{Elasticsearch}  \label{Elasticsearch}
The data collected from the dark web was store in Elasticsearch. ES is a distributed, open source search engine \cite{elasticSearch} and offers a fast full-text search. Another benefit of ES are documentation and supported tools, such as Logstash \cite{logstash} for processing of data or Kibana \cite{kibana} for the visualization of data. ES can be used as a NoSQL database. Such a database consists of indexes, documents and fields as opposed to tables, rows and columns of SQL databases.  One of the advantages of NoSQL databases is scalability, therefore they are suited for big amounts of data. 

\section{The data set} \label{dataSet}
The dark web pages were acquired via web scraping. The scraped pages belonged to two different networks - I2P  and Tor. The total number of pages was 221,844. Of those pages 212,851 were Tor pages and 8,993 I2P pages. The total number of unique domains was 5,178 of which 4,912 were from the Tor network and 266 were from the I2P network. 

The fields of a page entry will be described in the following list~\ref{DWDataList}. Each description will include a concrete example from the database. Fields beginning with an underscore are assigned to every document implicitly.
 \label{DWDataList}
\begin {description}
	\item[\_id] is a unique identifier. For example\\ \textit{2d622b6fba6f203d790fedbb4f47963e2366c7fd}.
	\item[\_index] informs about the collection the document belongs to. For example \textit{tor}. 
	\item[\_type] determines the type of the document. For example \textit{\_doc}.
	\item[content] is the actual content of the page. For example \textit{Purple Kush – 10g – WackyWeed Menu    * Home   * Contact us   * About us }.
	\item[content\_type] describes the type of the content. For example \textit{text/html; charset=UTF-8}.
	\item[domain] of the url address. For example \textit{wacky2yx73r2bjys.onion}.
	\item[h1] is the text with the h1 style. For example \textit{Purple Kush – 10g}.
	\item[links] to other pages this page links to. For example \textit{\{\\
  "link": "http://wacky2yx73r2bjys.onion/",\\
  "link\_name": "Home"\\
\}}.
	\item[raw\_text] is similar to \textit{content}. It additionally may contain formatting elements such as \textit{\textbackslash n}. For example \textit{Purple Kush – 10g – WackyWeed Menu\textbackslash n\textbackslash n  * Home\textbackslash n  * Contact us\textbackslash n  * About us\textbackslash n }.
	\item[raw\_title] is similar to title. It additionally may contain formatting elements. For example \textit{Purple Kush – 10g – WackyWeed}
	\item[raw\_url] is the same as \textit{url}.
	\item[title] of the page. For example \textit{Purple Kush – 10g – WackyWeed}.
	\item[updated\_on] depicts the time when the document in the database was last updated. For example \textit{2019-10-22T19:41:09}.
	\item[url] address of the page. For example \\ \textit{http://wacky2yx73r2bjys.onion/?product=purple-kush-10g}.
\end{description}